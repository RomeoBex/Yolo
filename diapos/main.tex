\documentclass{beamer}
\usepackage{graphicx}  % Nécessaire pour inclure des images





\title{You Only Look Once (YOLO)}
\author{BEX Roméo, RIVALDI Tristan, LAMURE Maxence}
\institute{Université de Montpellier \\
Master 2 Statistiques et Sciences des Données}
\date{2024 - 2025}

\begin{document}

% Page de titre
\begin{frame}
    \titlepage
    %\vspace{1cm}  % Espace entre le titre et le logo
    \begin{center}
        \includegraphics[width=0.7\textwidth]{logo.png} % Spécifiez la taille de l'image (20% de la largeur de la page)
    \end{center}
\end{frame}


% Table des matières
\begin{frame}{Table des matières}
    \tableofcontents
\end{frame}

% Introduction
\section{Introduction}

\begin{frame}{Introduction}
    La \textbf{classification d'images} consiste à associer des étiquettes à une image en fonction de son contenu visuel. C'est un problème clé de la \textit{vision par ordinateur}, souvent résolu via l'apprentissage supervisé. Applications courantes :
    \begin{itemize}
        \item \textbf{Reconnaissance faciale}
        \item \textbf{Voitures autonomes}
    \end{itemize}
\end{frame}



% Slide avec image sans boîte
\begin{frame}{Exemple d'image}
    \begin{figure}
        \centering
        \includegraphics[width=0.5\linewidth]{IMG_7995.jpeg}
        \caption{Image d'entrée.}
    \end{figure}
\end{frame}

% Slide avec image avec boîte
\begin{frame}{Image avec Boîtes Englobantes}
    \begin{figure}
        \centering
        \includegraphics[width=0.5\linewidth]{image_result.jpeg}
        \caption{Image avec boîtes englobantes prédites par YOLOv8.}
    \end{figure}
\end{frame}

% Méthodologie

\section{Méthodologie}
\begin{frame}{Formulation du problème}
    \begin{itemize}
        \item YOLO redimensionne puis divise l'image en une grille $S \times S$.\pause
        \item Chaque cellule prédit des \textbf{boîtes englobantes} et des \textbf{probabilités de classes}.\pause
        \item YOLO utilise une \textbf{régression directe}, prédisant en une seule étape les coordonnées des objets et leurs classes, ce qui permet un traitement rapide.
    \end{itemize}
\end{frame}

% Structure du réseau
\section{Structure du réseau}
\begin{frame}{Structure du réseau YOLO}
    \begin{itemize}
        \item Basé sur un réseau de convolution profond (24 couches + 2 couches entièrement connectées).\pause
        \item Sortie : tenseur de taille \(S \times S \times (B \times 5 + C)\). (\(C = 80\) avec notre modèle)\pause
        \item \(B\) : nombre de boîtes englobantes que chaque cellule de la grille prédit.\pause
        \item 5 : Chaque boîte englobante est décrite par 5 paramètres : \((x, y, w, h, \text{score de confiance})\).\pause
    \end{itemize}
    \begin{figure}
        \centering
        \includegraphics[width=1\linewidth]{architecture.png}
        \caption{Architecture du réseau YOLO.}
    \end{figure}
\end{frame}

% Fonction de perte

\begin{frame}{Fonction de perte}
    La fonction de perte de YOLO prend en compte les erreurs de localisation et de classification. 
    Elle utilise deux hyperparamètres \textcolor{red}{$\lambda_{\text{coord}}$} et \textcolor{blue}{$\lambda_{\text{noobj}}$} pour pondérer les différentes parties de la perte :\pause

    \begin{equation}
    \textcolor{red}{\lambda_{\text{coord}}} \sum_{i=0}^{S^2} \sum_{j=0}^{B} 1^{\text{obj}}_{ij} 
    \underbrace{\left[(x_i - \hat{x}_i)^2 + (y_i - \hat{y}_i)^2\right]}_{\text{Erreur de localisation}}
    \end{equation}\pause

    \begin{equation}
    + \textcolor{red}{\lambda_{\text{coord}}} \sum_{i=0}^{S^2} \sum_{j=0}^{B} 1^{\text{obj}}_{ij}
    \underbrace{\left[\left(\sqrt{w_i} - \sqrt{\hat{w}_i}\right)^2 + \left(\sqrt{h_i} - \sqrt{\hat{h}_i}\right)^2 \right]}_{\text{Erreur de taille (largeur/hauteur des boîtes)}}
    \end{equation}\pause

    \begin{equation}
    + \sum_{i=0}^{S^2} \sum_{j=0}^{B} 1^{\text{obj}}_{ij}
    \underbrace{\left(C_i - \hat{C}_i\right)^2}_{\text{Erreur avec objet}}
    + \textcolor{blue}{\lambda_{\text{noobj}}} \sum_{i=0}^{S^2} \sum_{j=0}^{B} 1^{\text{noobj}}_{ij}
    \underbrace{\left(C_i - \hat{C}_i\right)^2}_{\text{Erreur sans objet}}
    \end{equation}\pause
\end{frame}
\begin{frame}{Suite}
    \begin{equation}
    + \sum_{i=0}^{S^2} 1^{\text{obj}}_i \sum_{c \in \text{classes}} 
    \underbrace{\left(p_i(c) - \hat{p}_i(c)\right)^2}_{\text{Erreur de classification}}
    \end{equation}
\end{frame}


% Comparaison des Méthodes
\section{Comparaison des Méthodes}
\begin{frame}{Comparaison avec d'autres méthodes}
    \begin{figure}
        \centering
        \includegraphics[width=0.8\linewidth]{graphe.png}
        \caption{Courbes précision-rappel comparant YOLO avec d'autres méthodes.}
    \end{figure}
\end{frame}


% Limites et Faiblesses

\section{Limites et Faiblesses}
\begin{frame}{Limites et Faiblesses}
    \begin{itemize}
        \item Problèmes de localisation pour les petits objets.\pause
        \item Sensibilité aux petites boîtes, impactant l'IoU (précision des boîtes).\pause
    \end{itemize}

    \begin{figure}
        \centering
        \includegraphics[width=0.8\linewidth]{image_limite.jpeg}
        \caption{Limites de YOLOv8 sur une scène complexe.}
    \end{figure}
\end{frame}


\begin{frame}{Limites et Faiblesses - Exemple Supplémentaire}
    \begin{itemize}
        \item Difficulté à généraliser à des objets aux proportions inhabituelles.\pause
    \end{itemize}
    \begin{figure}
        \centering
        \includegraphics[width=0.4\linewidth]{image_limite2.jpeg}\pause
        \hspace{0.5cm}
        \includegraphics[width=0.4\linewidth]{image_limite3.jpeg}
    \end{figure}
\end{frame}





\section{Conclusion}
\begin{frame}{Conclusion}
    \begin{itemize}
        \item YOLO simplifie la détection d'objets en intégrant toutes les étapes en une seule.
        \item Avantages : rapidité, simplicité d'utilisation.
        \item Limites : localisation des petits objets, boîtes noires 
        \item Applications : robotique, surveillance,...
    \end{itemize}
\end{frame}




% Slide finale
\begin{frame}{Questions ?}
    \centering
    Merci pour votre attention ! \\
    Avez-vous des questions ?
\end{frame}

\end{document}

