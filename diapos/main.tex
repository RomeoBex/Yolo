\documentclass{beamer}
\usepackage{graphicx}  % Nécessaire pour inclure des images





\title{You Only Look Once (YOLO)}
\author{BEX Roméo, RIVALDI Tristan, LAMURE Maxence}
\institute{Université de Montpellier \\
Master 2 Statistiques et Sciences des Données}
\date{2024 - 2025}

\begin{document}

% Page de titre
\begin{frame}
    \titlepage
    %\vspace{1cm}  % Espace entre le titre et le logo
    \begin{center}
        \includegraphics[width=0.7\textwidth]{logo.png} % Spécifiez la taille de l'image (20% de la largeur de la page)
    \end{center}
\end{frame}


% Table des matières
\begin{frame}{Table des matières}
    \tableofcontents
\end{frame}

% Introduction
\section{Introduction}

\begin{frame}{Introduction}
    La \textbf{classification d'images} consiste à associer des étiquettes à une image en fonction de son contenu visuel. C'est un problème clé de la \textit{vision par ordinateur}, souvent résolu via l'apprentissage supervisé. Applications courantes :
    \begin{itemize}
        \item \textbf{Reconnaissance faciale}
        \item \textbf{Voitures autonomes}
    \end{itemize}
\end{frame}



% Slide avec image sans boîte
\begin{frame}{Image sans Boîte}
    \begin{figure}
        \centering
        \includegraphics[width=0.5\linewidth]{IMG_7995.jpeg}
        \caption{Image d'entrée.}
    \end{figure}
\end{frame}

% Slide avec image avec boîte
\begin{frame}{Image avec Boîtes Englobantes}
    \begin{figure}
        \centering
        \includegraphics[width=0.5\linewidth]{image_result.jpeg}
        \caption{Image avec boîtes englobantes prédites par YOLOv8.}
    \end{figure}
\end{frame}

% Méthodologie

\section{Méthodologie}
\begin{frame}{Formulation du problème}
    \begin{itemize}
        \item YOLO divise l'image en une grille $S \times S$.
        \item Chaque cellule prédit des \textbf{boîtes englobantes} et des \textbf{probabilités de classes}.
        \item YOLO utilise une \textbf{régression directe}, prédisant en une seule étape les coordonnées des objets et leurs classes, ce qui permet un traitement rapide.
    \end{itemize}
\end{frame}

% Structure du réseau
\begin{frame}{Structure du réseau YOLO}
    \begin{itemize}
        \item Basé sur un réseau de convolution profond (24 couches + 2 couches entièrement connectées).
        \item Traitement de l'image entière en un seul passage.
        \item Sortie : tenseur de taille S×S×(B×5+C).
    \end{itemize}
    
    \begin{figure}
        \centering
        \includegraphics[width=1\linewidth]{architecture.png}
        \caption{Architecture du réseau YOLO.}
    \end{figure}
\end{frame}

% Fonction de perte - Partie 1
\begin{frame}{Fonction de perte}
    La fonction de perte de YOLO prend en compte les erreurs de localisation et de classification. 
    Elle utilise deux hyperparamètres $\lambda_{\text{coord}}$ et $\lambda_{\text{noobj}}$ pour pondérer les différentes parties de la perte :
    
    \begin{equation}
    \lambda_{\text{coord}} \sum_{i=0}^{S^2} \sum_{j=0}^{B} 1^{\text{obj}}_{ij} \left[(x_i - \hat{x}_i)^2 + (y_i - \hat{y}_i)^2\right]
    \end{equation}

    \begin{equation}
    + \lambda_{\text{coord}} \sum_{i=0}^{S^2} \sum_{j=0}^{B} 1^{\text{obj}}_{ij} \left[\left(\sqrt{w_i} - \sqrt{\hat{w}_i}\right)^2 + \left(\sqrt{h_i} - \sqrt{\hat{h}_i}\right)^2 \right]
    \end{equation}
    
    \begin{equation}
    + \sum_{i=0}^{S^2} \sum_{j=0}^{B} 1^{\text{obj}}_{ij} \left(C_i - \hat{C}_i\right)^2 + \lambda_{\text{noobj}} \sum_{i=0}^{S^2} \sum_{j=0}^{B} 1^{\text{noobj}}_{ij} \left(C_i - \hat{C}_i\right)^2
    \end{equation}
    
    Où :
    \begin{itemize}
        \item $1^{\text{obj}}_{ij}$ est un indicateur qui vaut 1 si l'objet est présent dans la cellule de la grille $i$ et 0 sinon.
        \item $x_i, y_i$ sont les coordonnées du centre de la boîte prédite.
    \end{itemize}
\end{frame}

% Fonction de perte - Partie 2
\begin{frame}{Fonction de perte (suite)}
    Suite des explications :
    \begin{itemize}
        \item $C_i$ est le score de confiance de la boîte.
        \item $\lambda_{\text{coord}} = 5$ et $\lambda_{\text{noobj}} = 0.5$ sont les pondérations des différents termes de la perte.
        \item Les termes en $\lambda_{\text{coord}}$ pondèrent l'erreur de localisation (coordonnées et dimensions des boîtes).
        \item Les termes en $\lambda_{\text{noobj}}$ réduisent l'impact des fausses prédictions dans les cellules sans objet.
    \end{itemize}
\end{frame}

% Comparaison des Méthodes
\section{Comparaison des Méthodes}
\begin{frame}{Comparaison avec d'autres méthodes}
    \begin{itemize}
        \item YOLO est comparé à Faster R-CNN et Fast R-CNN.
        \item YOLO est plus rapide avec un bon équilibre entre vitesse et précision.
    \end{itemize}
    \begin{table}[]
    \centering
    \begin{tabular}{|c|c|c|}
    \hline
    Méthode       & mAP (\%) & FPS \\ \hline
    Fast R-CNN    & 70.0     & 0.5 \\ \hline
    Faster R-CNN  & 73.2     & 7   \\ \hline
    YOLO          & 63.4     & 45  \\ \hline
    Fast YOLO     & 52.7     & 155 \\ \hline
    \end{tabular}
    \caption{Comparaison des méthodes sur PASCAL VOC 2007}
    \end{table}
\end{frame}

% Limites et Faiblesses

\section{Limites et Faiblesses}
\begin{frame}{Limites et Faiblesses}
    \begin{itemize}
        \item Problèmes de localisation pour les petits objets.
        \item Sensibilité aux petites boîtes, impactant l'IoU (précision des boîtes).
        \item Difficulté à généraliser à des objets aux proportions inhabituelles.
    \end{itemize}

    \begin{figure}
        \centering
        \includegraphics[width=0.8\linewidth]{image_limite.jpeg}
        \caption{Limites de YOLOv8 sur une scène complexe.}
    \end{figure}
\end{frame}

\begin{frame}{Limites et Faiblesses - Exemple Supplémentaire}
    \begin{figure}
        \centering
        \includegraphics[width=0.9\linewidth]{image_limite2.jpeg}
    \end{figure}
\end{frame}



\section{Conclusion}
\begin{frame}{Conclusion}
    \begin{itemize}
        \item YOLO simplifie les processus classiques de détection d'objets en intégrant toutes les étapes en une seule.
        \item Avantages: Rapidité et facilité d'utilisation.
        \item Limitations: Localisation imprécise des petits objets.
        \item Applications futures: robotique, surveillance, réalité augmentée.
    \end{itemize}
\end{frame}

\section{Conclusion}
\begin{frame}{Conclusion}
    \begin{itemize}
        \item YOLO simplifie la détection d'objets en intégrant toutes les étapes en une seule.
        \item Avantages : rapidité, simplicité d'utilisation.
        \item Limites : localisation des petits objets.
        \item Applications : robotique, surveillance,...
    \end{itemize}
\end{frame}




% Slide finale
\begin{frame}{Questions ?}
    \centering
    Merci pour votre attention ! \\
    Des questions ?
\end{frame}

\end{document}
